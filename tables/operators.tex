  \begin{tabular}{cC{9cm}}
    \toprule
    Lagrangian term & Notation\\
    \midrule
    & \multirow{10}{*}{\parbox{9cm}{We follow the notation described
    in reference~\cite{Alloul2014}:
    $B_\mu$, $W^k_\mu$, and $G^a_\mu$ ($g^\prime$, $g$, and $g_s$) refer to the $U(1)_Y$, $SU(2)_L$, and
    $SU(3)_c$ gauge vector fields (coupling constants), respectively. The corresponding $B_{\mu\nu}$, $W^k_{\mu\nu}$, and $G^a_{\mu\nu}$ field tensors are
    defined in Equation (2.8) of reference~\cite{Alloul2014}. The
    left-handed $Q_L=\begin{psmallmatrix} u_L\\d_L \end{psmallmatrix}$ and right-handed
    up-type $u_R$ quark 
    fields refer to the three generations in the context of the \eightTeV
    analysis and to the third generation in the context of the \thirteenTeV
    analysis. The Higgs doublet is referred to as $\Phi$. The Hermitian derivative operators ${\overleftrightarrow D}_\mu$ are
    defined as
      $\Phi^\dag {\overleftrightarrow D}_\mu \Phi = \Phi^\dag D^\mu \Phi - D_\mu\Phi^\dag \Phi$.
    The Higgs quartic coupling and vacuum expectation value is given by
    $\lambda$ and $v$, respectively. The $3 \cross 3$ Yukawa coupling matrix in
    flavor space is
    given by $y_u$. The generators of $SU(2)$ are given by
    $T_{2K}=\frac{\sigma_k}{2}$, where $\sigma_k$ are the Pauli matrices.
}}\\
      % B_{\mu\nu} =&\ \partial_\mu B_\nu - \partial_\nu B_\mu \ ,\\
      % W^k_{\mu\nu} =&\ \partial_\mu W^k_\nu - \partial_\nu W^k_\mu + g \epsilon_{ij}{}^k \ W^i_\mu W^j_\nu\ ,\\
      % G^a_{\mu\nu} =&\ \partial_\mu G^a_\nu - \partial_\nu G^a_\mu + g_s f_{bc}{}^a\ G^b_\mu G^c_\nu\ ,\\
      % D_\rho W^k_{\mu\nu} = &\ \partial_\mu\partial_\rho W^k_\nu - \partial_\nu\partial_\rho W^k_\mu +
      % g \epsilon_{ij}{}^k \partial_\rho\big[W_\mu^i W_\nu^j\big] +
      % g \epsilon_{ij}{}^k W_\rho^i \big[\partial_\mu W_\nu^j - \partial_\nu W_\mu^j\big]\\ &\quad + 
      % g^2  W_{\rho i}\big[W_\nu^i W_\mu^k - W_\mu^i W_\nu^k\big] \ , \\
      % D_\mu\Phi =&\ \partial_\mu \Phi - \frac12 i g' B_\mu \Phi -  i g T_{2k} W_\mu^k \Phi \end{align}}
    $\frac{\cH}{2 v^2}\partial^\mu\big[\Phi^\dag \Phi\big] \partial_\mu \big[ \Phi^\dagger \Phi \big]$ & \\
    $\frac{\cHQ}{v^2}\big[\bar Q_L \gamma^\mu Q_L\big] \big[ \Phi^\dag{\overleftrightarrow D}_\mu \Phi\big]$ & \\
    $\frac{4 i \cpHQ}{v^2}\big[\bar Q_L \gamma^\mu T_{2k} Q_L\big]  \big[\Phi^\dag T^k_2 {\overleftrightarrow D}_\mu \Phi\big]$ & \\
    $\frac{i \cHu}{v^2}\big[\bar u_R \gamma^\mu u_R\big] \big[ \Phi^\dag{\overleftrightarrow D}_\mu \Phi\big]$ & \\
    $\frac{2 g' \cuB}{\PmW^2}y_u\ \Phi^\dag \cdot {\bar Q}_L \gamma^{\mu\nu} u_R \  B_{\mu\nu}$ & \\
    $\frac{4 g \cuW}{\PmW^2}y_u\ \Phi^\dag \cdot \big({\bar Q}_L T_{2k}\big) \gamma^{\mu\nu} u_R  \ W_{\mu\nu}^k$ & \\
    $\frac{g^3 \cthreeW}{\PmW^2}\epsilon_{ijk} W_{\mu\nu}^i W^\nu{}^j_\rho W^{\rho\mu k}$ & \\
    $\frac{g_s^3 \tcthreeG}{\PmW^2} f_{abc} G_{\mu\nu}^a G^\nu{}^b_\rho {\widetilde G}^{\rho\mu c}$ & \\
    $\frac{g_s^3\ \cthreeG}{\PmW^2} f_{abc} G_{\mu\nu}^a G^\nu{}^b_\rho G^{\rho\mu c}$ & \\
    $\frac{\ctwoG}{\PmW^2} D^\mu G_{\mu\nu}^a D_\rho G^{\rho\nu}_a$ & \\
  % \parbox{3cm}{\begin{equation}a=x\end{equation}}
    \bottomrule
  \end{tabular}
% and the $SU(2)$ invariant products
% \be
%   Q_L\cdot\Phi = \epsilon_{ij}\ Q_L^i\ \Phi^j
%   \quad\text{and}\quad
%   \Phi^\dag\cdot \bar Q_L = \epsilon^{ij}\ \Phi^\dag_i\ \bar Q_{Lj} \ ,
% \ee
% the rank-two antisymmetric tensors being defined by $\epsilon_{12}=1$ and
% $\epsilon^{12}=-1$. Finally,
% our conventions for the gauge-covariant derivatives and the gauge field strength tensors are
% \be\bsp
% \esp\label{eq:covder}\ee
% $\epsilon_{ij}{}^k$ and $f_{ab}{}^c$ being the structure constants of $SU(2)$ and
% $SU(3)$.
% The Lagrangian of Eq.~\eqref{eq:silh} can be supplemented by extra $CP$-violating operators,
% \be\label{eq:silhCPodd}\bsp
%   {\cal L}_{CP} = &\
%   + \frac{g'^2\  \tilde c_{\sss \gamma}}{\mW^2} \Phi^\dag \Phi B_{\mu\nu} {\widetilde B}^{\mu\nu}\\
%  &\
% \esp\ee
% where the dual field strength tensors are defined by
% \be
%   \widetilde B_{\mu\nu} = \frac12 \epsilon_{\mu\nu\rho\sigma} B^{\rho\sigma} \ , \quad
%   \widetilde W_{\mu\nu}^k = \frac12 \epsilon_{\mu\nu\rho\sigma} W^{\rho\sigma k} \ , \quad
%   \widetilde G_{\mu\nu}^a = \frac12 \epsilon_{\mu\nu\rho\sigma} G^{\rho\sigma a} \ .
% \ee

% \be\label{eq:lf1}\bsp
%   {\cal L}_{F_1} = &\ 
%   &\
%   + \frac{i \bar c_{\sss Hd}}{v^2} \big[\bar d_R \gamma^\mu d_R\big]  \big[ \Phi^\dag{\overleftrightarrow D}_\mu \Phi\big]\\
%   &\  -\bigg[
%     \frac{i \bar c_{\sss Hud}}{v^2} \big[\bar u_R \gamma^\mu d_R\big]  \big[ \Phi \cdot {\overleftrightarrow D}_\mu \Phi\big]
%    + {\rm h.c.} \bigg] \\
%   &\ +
%     \frac{i \bar c_{\sss HL}}{v^2}  \big[\bar L_L \gamma^\mu L_L\big] \big[ \Phi^\dag{\overleftrightarrow D}_\mu \Phi\big]
%   + \frac{4 i \bar c'_{\sss HL}}{v^2} \big[\bar L_L \gamma^\mu T_{2k} L_L\big]  \big[\Phi^\dag T^k_2 {\overleftrightarrow D}_\mu \Phi\big] \\
%   &\
%   + \frac{i \bar c_{\sss He}}{v^2} \big[\bar e_R \gamma^\mu e_R\big]  \big[ \Phi^\dag{\overleftrightarrow D}_\mu \Phi\big] \ ,
% \esp\ee
% whilst the fourth term of this Lagrangian addresses the interactions of a quark or lepton pair and one single Higgs field
% and a gauge boson,
% \be\label{eq:lf2}\bsp
%   {\cal L}_{F_2} = &\  \bigg[
%   &\quad
%   - \frac{4 g_s\ \bar c_{\sss uG}}{\mW^2} y_u\ \Phi^\dag \cdot {\bar Q}_L \gamma^{\mu\nu} T_a u_R G_{\mu\nu}^a 
%   + \frac{2 g'\ \bar c_{\sss dB}}{\mW^2}  y_d\ \Phi {\bar Q}_L \gamma^{\mu\nu} d_R \  B_{\mu\nu}\\
%  &\quad
%   + \frac{4 g\ \bar c_{\sss dW}}{\mW^2}   y_d\ \Phi \big({\bar Q}_L T_{2k}\big) \gamma^{\mu\nu} d_R  \ W_{\mu\nu}^k
%   + \frac{4 g_s\ \bar c_{\sss dG}}{\mW^2} y_d\ \Phi {\bar Q}_L \gamma^{\mu\nu} T_a d_R G_{\mu\nu}^a \\
%  &\quad
%   + \frac{2 g'\ \bar c_{\sss eB}}{\mW^2}  y_\ell\ \Phi {\bar L}_L \gamma^{\mu\nu} e_R \  B_{\mu\nu}
%   + \frac{4 g\ \bar c_{\sss eW}}{\mW^2}   y_\ell\ \Phi \big({\bar L}_L T_{2k}\big) \gamma^{\mu\nu} e_R  \ W_{\mu\nu}^k 
%  +  {\rm h.c.} \bigg]\ .
% \esp\ee
% In this expression, the matrices $T_a$ are the generators of the $SU(3)$ group in the fundamental representation
% and the $\gamma^{\mu\nu}$ quantities, defined by
% \be
%   \gamma^{\mu\nu} = \frac{i}{4} \Big[\gamma^\mu,\gamma^\nu\Big] \ ,
% \ee
% are the generators of the Lorentz algebra in the (four-component) spinorial representation.
% In the most general case, the Wilson coefficients $\bar c_i$ related to the fermionic operators included in
% the Lagrangians ${\cal L}_{F_i}$ are tensors in flavor space and complex quantities.

% The last term of Eq.~\eqref{eq:effL} refers to operators not directly connected to Higgs physics, but that may
% be important as affecting the gauge sector and possibly modifying the
% gauge boson self-energies and self-interactions,
% \be\bsp
%   {\cal L}_{G} = &\
%   + \frac{\bar c_{\sss 2W}}{\mW^2} D^\mu W_{\mu\nu}^k D_\rho W^{\rho\nu}_k \\
%  &\
%   + \frac{\bar c_{\sss 2B}}{\mW^2} \partial^\mu B_{\mu\nu} \partial_\rho B^{\rho\nu}
% \esp\label{eq:lagG}\ee
% with
% \be\bsp
%   D_\rho G^a_{\mu\nu} = &\ \partial_\mu\partial_\rho G^a_\nu - \partial_\nu\partial_\rho G^a_\mu +
%     g_s f_{bc}{}^a \partial_\rho\big[G_\mu^b G_\nu^c\big] +
%     g_s f_{bc}{}^a G_\rho^b \big[\partial_\mu G_\nu^c - \partial_\nu G_\mu^c\big]\\ &\quad + 
%     g_s^2  G_{\rho b}\big[G_\nu^b G_\mu^a - G_\mu^b G_\nu^a\big] \ ,
% \esp\ee
% recalling that $D_\mu W^{\nu\rho}_k$ has been defined in Eq.~\eqref{eq:covder}.

% Finally, 22 independent
% baryon and lepton number conserving four-fermion operators are also allowed by gauge invariance. Since they
% have no effects on Higgs physics, at least at the leading order and in the context
% of the LHC phenomenology, we omit them from the present manuscript, as already
% above-mentioned. Moreover, as indicated in
% Ref.~\cite{Contino:2013kra}, two of the 39 operators that have been introduced are redundant and can be removed through
% \be\bsp
%   {\cal O}_{\sss W} =&\ -2 {\cal O}_{\sss H} + \frac{4}{v^2} \Phi^\dag \Phi D^\mu\Phi^\dag D_\mu\Phi + {\cal O}'_{\sss HQ} + 
%     {\cal O}'_{\sss HL} \ , \\
%   {\cal O}_{\sss B} =&\ 2 \tan^2\theta_{\sss W} \Big[ \sum_\psi Y_\psi {\cal O}_{\sss H\psi} - {\cal O}_{\sss T} \Big] \ ,
% \esp\ee
% where we sum over the whole chiral content of the theory and $\theta_{\sss W}$ stands for the weak mixing angle
% (see Eq.~\eqref{eq:weakmix1} and Eq.~\eqref{eq:weakmix2} in Section~\ref{sec:massbasis}).
% We however include them in our effective field
% theory description as according to the specific effect of interest, one choice for an operator basis
% may be more suitable than another.

