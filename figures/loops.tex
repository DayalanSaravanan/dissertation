\begin{subfigure}{0.5\linewidth}
  \centering
  \feynmandiagram [horizontal=l2 to h] {
    g1 -- [gluon] l1 -- [anti fermion, edge label={\Ptop, \Pb}] l2 -- [scalar, edge label=\PH] h,
    g2 -- [gluon] l3 -- [fermion] l2,
    l3 -- [anti fermion] l1,
    g1 -- [draw=none] g2,
  };
  \caption{}
  \label{sfig:ggf}
\end{subfigure}%
\begin{subfigure}{0.5\linewidth}
  \centering
  \feynmandiagram [horizontal=Z to l2] {
    q1 -- [anti fermion, edge label={\Pb}] l1 -- [anti fermion, edge label={\Ptop}] l2 -- [boson, edge label=\PZ] Z,
    q2 -- [fermion, edge label={\(\Pbbar\)}] l3 -- [fermion, edge label={\Ptopbar}] l2,
    l3 -- [boson, edge label={\PW}] l1,
    q1 -- [draw=none] q2,
  };
  \caption{}
  \label{sfig:tz}
\end{subfigure}
% TODO second diagram is reversed (antifermion should be fermion etc)
