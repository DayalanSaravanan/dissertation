\chapter{Object reconstruction and selection}
\label{chap:objects}
The CMS subdetectors produce raw electrical signals, from which \emph{physics
objects} are inferred as software representations of particle observables. This
process is called \emph{reconstruction} and is described
in~\cref{sec:particle-flow}. In~\cref{sec:object-selection}, the selection
criteria for physics objects used in this work are discussed. Except where
noted, features are common to both the \eightTeV measurement (presented
in~\cref{chap:8-TeV}) and the \thirteenTeV measurement (presented
in~\cref{chap:13-TeV}).

\section{Particle flow}
\label{sec:particle-flow}
CMS uses a reconstruction approach called \emph{Particle Flow} (PF), which takes
advantage of information provided by each of the CMS subdetectors, as well as
the correlations between these observations to construct a global picture of
each event. The following description of the CMS PF algorithms is based on
reference~\cite{Sirunyan:2017ulk}.

The reconstructed origin of a set of physics objects is called a \emph{vertex}.
An example of a vertex is the origin of the physics objects emerging from a
proton-proton collision. As shown in~\cref{fig:luminosity}, each bunch crossing
results in multiple proton-proton collisions. The majority of these are pileup.
The vertex for which $\sum\pT^2$ is highest is most likely to be associated with
the production of heavy particles, and is chosen to be the \emph{primary vertex}
(PV). For the \eightTeV analysis, the sum is taken over all charged particles
that were used in the vertex reconstruction. For the \thirteenTeV analysis, the
sum is taken over all charged leptons and jets formed from charged tracks used
in the vertex reconstruction, in addition to the \pTmiss, which is computed as
the magnitude vector sum of \pT from all PF candidates.

An iterative approach is used~\cite{Chatrchyan:2014fea} to reconstruct particle
trajectories in the inner tracker. Each iteration begins with seeds generated
from a few tracker hits. Next, trajectories are built by adding hits from
successive track layers. Finally, track-fitting using a Kalman
filter~\cite{Fruhwirth:1987fm} is performed to calculate the track origin and
direction, and the \pT. In each iteration, tracks are subject to quality
requirements on the seeds, $\chi^2$ fit, vertex compatibility, \pT, $|\eta|$,
and number of hits. After each iteration, hits that have been assigned to tracks
are removed from the available hit pool. The first iteration applies stringent
criteria, which produces a low rate of misreconstructed tracks with moderate
tracking efficiency. Over 10 total iterations, the criteria are progressively
relaxed as the algorithms become more complex and time-consuming. After the size
of the hit pool has been sufficiently reduced, later iterations can perform more
computationally intensive searches for displaced tracks with secondary vertices
outside the interaction region (mainly due to photon conversions or nuclear
interactions in the tracker material, b jets, $\Delta$ decays, \etc)

Most electrons radiate bremsstrahlung photons due to the coulomb field of atomic
nuclei as they pass through the tracker. These photons are emitted in a
characteristic arc due to the azimuthal bending of the electron track in the
magnetic field. To account for these interactions, tracks likely to be electrons
are selected after the iterative procedure described above based on their number
of hits and $\chi^2$ fit. The track-fitting is then repeated with a Gaussian Sum
Filter (GSF)~\cite{0954-3899-31-9-N01}, which approximates the electron energy
loss as a sum of Gaussian distributions instead of a single Gaussian, resulting
in an improved momentum resolution.

The determination of particle energy measured in the calorimeters is
accomplished with a \emph{clustering} algorithm, which takes as input a map of
individual energy deposits and outputs groupings of energy deposits that can be
associated with candidate physics objects. First, \emph{cluster seeds} are
determined by identifying calorimeter cells with energies exceeding a predefined
threshold. Next, clusters are built up by adding adjacent cells with energy
deposits exceeding typical electronic noise levels.

After the PF elements have been assembled, a \emph{linking} algorithm evaluates
the degree of the compatibility between different subdetector elements that are
nearest neighbors in the $(\phi, \eta)$ plane to determine which are likely to
be associated with the same particle. Tracks are extrapolated from the last hit
measured in the tracker to the PS and subsequently to the ECAL and HCAL. If the
extrapolation falls within the cluster boundaries, a link is established. ECAL
clusters that are consistent with the extrapolation of tangents from points in
which inner tracks intersect with tracker layers are linked as possible
bremsstrahlung photons. Photons have a substantial probability to pair produce
in the tracker material, so a similar process is used to identify possible
photon conversions. If the $(\phi, \eta)$ cluster position in the ECAL is within
the cluster envelope in the HCAL, a link is established. \emph{Global muons} are
formed if tracks in the inner tracker and tracks in the muon system can be
linked by interpolating towards each other with an acceptable $\chi^2$.
\emph{Tracker muons} are formed if the extrapolation out to the muon track from
the inner track is sufficiently consistent. The highest-quality links are used
to form sets of PF elements that are linked, or have common elements that are
linked. These sets are called \emph{blocks}.

Following linking, PF elements are assigned particle identifications. To reduce
combinatorics, elements are removed from blocks after assignment. Muons are
assigned first because of the high purity with which they can be reconstructed,
and they are required to satisfy any of three sets of criteria. Particle
\emph{isolation} quantifies the degree to which a particle is separated from
others. Isolated muons are identified if the sum of the track \pT and transverse
energy of calorimeter deposits in a cone with $R=0.3$ does not exceed
\SI{10}{\percent} of the muon \pT. Nonisolated muons are identified if they pass
a more stringent selection~\cite{Chatrchyan:2013sba} optimized for muons within
jets. Finally, muons that fail the tight selection because of problems with the
inner track reconstruction, but which have a sufficiently high-quality fit in
the muon system, are assigned as PF muons and removed from their blocks.

Because of the tendency for electrons to emit bremsstrahlung photons, and for
photons to pair produce electrons in the tracker material, electrons and
isolated photons are reconstructed together, following muon reconstruction.
Electron candidates are seeded with tracks in the inner tracker linked to
sufficiently large energy deposits in the ECAL. Photon candidates are seeded
from similar energy deposits that have no linked corresponding inner tracks.
Electron and photon candidates must meet requirements on the distribution of
energy deposits in the calorimeters. Photons additionally must be isolated from
other tracks and energy deposits. A multivariate discriminator based on track
variables and energy distribution patterns is used to select electrons.

% http://ific.uv.es/tical/Publications/InternalNotes/Nota_Atlfast.pdf
After muons, electrons, and isolated photons have been classified and removed
from their blocks, hadrons and nonisolated photons are identified. The majority
of jet energy is carried by charged hadrons, then photons, and finally neutral
hadrons. For this reason, within the tracker acceptance, ECAL clusters that are
not linked to inner tracks are classified as photons, and HCAL clusters that are
not linked to inner tracks are classified as neutral hadrons. If the calorimeter
clusters are linked to tracks, each track is assumed to give rise to a charged
hadron. If the calibrated calorimeter energy is greater than the total track
momenta by more than what is expected from the energy resolution to measure
charged hadrons, the surplus is assumed to be carried by photons and neutral
hadrons: up to the ECAL energy, the surplus is assigned as a photon, and any
energy above that is assigned as a neutral hadron. Outside of the tracker
acceptance, where charged and neutral hadrons cannot be distinguished, ECAL
clusters with no associated HCAL cluster are classified as photons, while ECAL
clusters that are linked to HCAL clusters are classified as arising from the
same hadronic shower. The response produced in the ECAL to photons and charged
hadrons with equivalent energy is different and must be corrected for after
particle identification has been made.

The anti-$k_t$ algorithm~\cite{1126-6708-2008-04-063} is used to identify which
PF candidate hadrons belong to the same jet. The quantity
\begin{eqnarray}
  d_{ij} = \frac{\Delta_{ij}^2}{R^2 \max(k_{ti}, k_{tj})}
\end{eqnarray}
is calculated for all possible candidate pairs $i,j$, where $R$ is a
characteristic radius parameter, $\Delta_{ij}^2 = (y_i - y_j)^2 +
(\phi_i-\phi_j)^2$, and $k_{ti}$, $y_i$, and $\phi_i$ are respectively the
transverse momentum, rapidity, and $\phi$-coordinate of candidate $i$. The pair
that minimizes $d_{ij}$ is replaced by a clustered candidate with the sum of
their four-momenta. If all possible $d_{ij} > 1/k_{ti}$ for any candidate, it is
promoted to a jet and removed from consideration. The process is repeated until
no candidates remain. If there are many soft particles and only one hard
particle within a radius $2R$, the minimum $d_{ij}$ will be dominated by the
momentum of the "seed" hard particle, which will accumulate soft particles,
resulting in a jet that is a circle around the hard particle in the $\phi-\eta$
plane of radius R. If there are two hard particles that are more than $R$ apart
but less than $2R$, there will be two jets, with more of the soft particles
naturally assigned to the higher-$k_t$ particle. If there are two hard particles
less than $R$ apart, they will be merged into a single jet.

Anti-$k_t$ jets with $\text{R}=0.5$ are used in the \eightTeV analysis. Because
the average jet energy increases with increasing center-of-mass energy,
resulting in narrower jets, the cone size was reduced to $\text{R}=0.4$ in the
\thirteenTeV analysis.

Because top quarks almost always decay to b quarks, identifying jets from b
quarks (called \emph{b-tagging}) is critical for selecting \ttW and \ttZ events.
Because the b quark is the lighter of the third generation quark doublet, it
decays via a generation-changing process to c or u quarks, which are
off-diagonal CKM matrix elements and therefore suppressed. This results in a
substantially longer b quark lifetime than expected for their mass. On the other
hand, it is short enough that decays still occur inside the detector. The long
lifetime of b hadrons in jets originating from the hadronization of b quarks
allows them to travel $\mathcal{O}(\SI{1}{mm})$ before decaying at a displaced
\emph{secondary vertex}. In the \eightTeV analysis, a Combined Secondary Vertex
(CSV) b-tagging algorithm was used~\cite{1748-0221-8-04-P04013}. The CSV
algorithm uses variables related to the distance from a track's point of origin
to the PV (the impact parameter, or IP) along with secondary vertex information
to produce a continuous score between zero and one that is assigned to each jet.
Higher scores indicate that the jet is more likely to have originated from a b
quark than a light-flavor quark or gluon. In the \thirteenTeV analysis,
improvements were made to allow for the inclusion of additional variables,
improving the tagging efficiency by several percentage points, and is referred
to as CSVv2~\cite{CMS-PAS-BTV-15-001}.

\section{Object selection}
\label{sec:object-selection}
Our analyses, like most on CMS, rely on the PF and jet clustering procedure
described above. Additional requirements, optimized for selecting \ttW and \ttZ
events, are discussed below.

\subsection{Leptons}
\label{ssec:leptons}
For the \ttW and \ttZ measurements, differentiating \emph{prompt} from
\emph{nonprompt} leptons is critical. Prompt leptons are muons or electrons
arising either directly from a W or Z boson parent, or from a tau lepton which
arise directly from a W or Z boson parent. Prompt leptons that have
misidentified charge are called \emph{charge flip} leptons. Nonprompt leptons
arise mainly from b hadron decays, misidentified jets, or photon conversions.

Prompt leptons are generally more isolated from other objects in the event than
nonprompt leptons, which are often produced in hadronic decays. To assess the
degree of isolation, the scalar sum of PF particles within a cone of $\Delta R =
\sqrt{\smash[b]{(\Delta \eta)^2 + (\Delta \phi)^2}} = 0.4$ (or $\Delta R = 0.3$
for electrons in the \thirteenTeV analysis) around the lepton direction is
calculated. To account for the charged component of pileup, the sum \pT of
charged particles not originating from the PV is subtracted. To account for the
neutral component, half of the charged PU contribution is subtracted in the
\eightTeV analysis. In the \thirteenTeV analysis, a $\rho(\eta)$ correction is
used as described in~\cref{sec:jet-clustering}. The ratio of this corrected sum
to the lepton \pT is called the relative isolation. Prompt leptons are also
characterized by low minimum displacement of the track from the vertex in the
transverse ($d_{xy}$) and longitudinal ($d_z$) planes, and by the ratio of the
three-dimensional impact parameter to its uncertainty ($\text{SIP}_\text{3D}$).
Electrons are additionally required to pass cuts on the score from an ElectronID
MVA classifier which combines discriminating power from a set of shower-shape,
track-cluster consistency, and track quality variables. This classifier was
re-optimized for the \thirteenTeV analysis.

Analysis-specific criteria, along with a description of the different criteria
categories used for event selection and data-driven background estimations, are
summarized below for the \eightTeV and \thirteenTeV analyses.

\subsubsection{\eightTeV}
The \eightTeV analysis imposes requirements related to the properties of the
nearest jet. A nearby jet with a high CSV value indicates that the lepton is
likely to be nonprompt, originating from b-hadron decay. Furthermore, the ratio
of lepton \pT to that of the nearest jet tends to be lower for such leptons.

Four sets of lepton selection criteria are defined: preselected, loose, tight,
and good charge. Preselected leptons, with criteria designed to select both
prompt (selected with $\sim \SI{100}{\percent}$ efficiency) and nonprompt
leptons, are used in the data-driven background estimations described
in~\cref{sec:8-modeling}. Tight leptons, used to select events in the SS
dilepton and three lepton channels, must pass more stringent criteria designed
to accept prompt and reject nonprompt leptons. The efficiency for accepting
prompt leptons passing the tight criteria is \SIrange{68}{98}{\percent} for
muons and \SIrange{49}{93}{\percent} for electrons, while \SI{80}{\percent} of
nonprompt leptons are rejected. Loose leptons, selected with a less stringent
set of criteria, are used to select events in the OS dilepton and four lepton
channels, where there are fewer nonprompt leptons. These cuts accept
\SIrange{93}{99}{\percent} of prompt muons and \SIrange{89}{96}{\percent} of
prompt electrons while rejecting $\sim \SI{50}{\percent}$ of nonprompt leptons.
In addition to the preselected, loose, or tight criteria, additional charge ID
requirements are imposed to reject charge flip leptons. This charge ID cut has
\SI{99}{\percent} efficiency for right charge muons and rejects $\sim
\SI{100}{\percent}$ of charge flip muons; for electrons, the efficiency ranges
from \SIrange{85}{100}{\percent}, while more than \SI{97}{\percent} of charge
flip electrons are rejected. These four sets of selection criteria are
summarized in~\cref{tab:8-TeV-leptons}.
\begin{table}
  \centering
  \caption[Lepton Selection Criteria (\eightTeV)]{
    Lepton Selection Criteria (\eightTeV)
  }
  \input{tables/eight-TeV/leptons}
  \label{tab:8-TeV-leptons}
\end{table}

\begin{table}
  \input{tables/thirteen-TeV/leptons}
\end{table}

\subsubsection{\thirteenTeV}
\label{subsec:13_lep_selection}
The \thirteenTeV analysis defines two sets of lepton selection criteria. Loose
leptons are used in the data-driven background estimations. Tight leptons are
used to select signal events. Additional quality criteria are imposed,
corresponding to the medium working point recommended by the muon Physics Object
Group (POG). These selection criteria are summarized
in~\cref{tab:13-TeV-leptons}.
% muon POG MediumID2016 WP: http://cms.cern.ch/iCMS/jsp/db_notes/noteInfo.jsp?cmsnoteid=CMS%20AN-2017/045


\subsection{Jets and missing energy}
\label{sec:jet-clustering}
Jet energy corrections (JECs), described in detail in
reference~\cite{1748-0221-12-02-P02014}, are required to calibrate the
reconstructed jet energy to match the true parton energy. To mitigate the
effects of pileup, charged hadrons that do not originate from the PV are removed
before jet clustering as described in reference~\cite{CMS-PAS-JME-14-001}. To
account for neutral hadrons from pileup, the average energy density as a
function of the pseudorapidity $\rho(\eta)$ is calculated for each event and,
for each jet, multiplied by the effective jet area and subtracted. Jets coming
from pileup vertices are removed using a multivariate
discriminator~\cite{CMS-PAS-JME-13-005} based on tracking information and jet
shape, with sensitivity driven by the ratio of the total \pT from charged
candidates that do not originate from the PV to the total \pT of all charged
candidates in the jet and the average spread in \pT among jet PF candidates.
Additional corrections, parameterized in $\eta$ and jet \pT, are applied to
account for residual differences between data and simulation, and the nonlinear
detector response to hadrons. To reject fake jets from misreconstruction and
instrumental noise, jets must have at least two PF constituents and the fraction
of their energy from the ECAL or HCAL deposits must not exceed
\SI{99}{\percent}.

Jets with $|\eta| < 2.4$ and $\pT >$ \SI{25}{~GeV} ($\pT >$ \SI{30}{~GeV}) are
selected for the \eightTeV (\thirteenTeV) analysis. To prevent double counting,
jets must be separated from the selected leptons by $\Delta R > 0.5$  for the
\eightTeV analysis and $\Delta R > 0.4$ for the \thirteenTeV analysis.

Loose and medium working points for the b-tagging output discriminant are used,
defined to operate with a $\approx$\SI{10}{\percent} and
$\approx$\SI{1}{\percent} chance, respectively to incorrectly tag light quark or
gluon jets (i.e., to \emph{mistag}). This approach corresponds to an efficiency
of $\approx$\SI{85}{\percent} to correctly tag b jets for the loose working
point and $\approx$\SI{70}{\percent} for the medium working point, depending on
the jet \pT and $\eta$. To satisfy the loose or medium working point,
$\text{CSV} > 0.244$ and $\text{CSV} > 0.679$ were required respectively for the
\eightTeV analysis. For the \thirteenTeV analysis, this requirement corresponded
to $\text{CSVv2} > 0.5426$ and $\text{CSVv2} > 0.8484$.

The \pTmiss is defined as the magnitude of the sum of the negative transverse
momenta of all of the PF particles. Because pileup can contribute to \pTmiss,
the \eightTeV analysis makes use of the variable \HTmiss, computed as the sum of
the negative transverse momenta of selected jets and leptons. The \HTmiss is
more robust than \pTmiss at the expense of reduced resolution.

